\begin{titlepage}
	\maketitle{}

	\begin{abstract}
		Ceci est un article traitant de l'usage du numérique pendant les phases de confinement de la Covid-19 ainsi que l'évolution possible de tout cet usage numérique.
		
		Cet article utilise de nombreux outils mis en place par un système corruptif impliquant de nombreux membres du parlement, ainsi que quelques uns du gouvernement.
		
		Contrairement aux anciennes accusations que nous portions dans le numéro 2031 de la revue \emph{Millenium}, qui fûrent démontées au tribunal par l'armée d'avocats qui a envoyé en prison pour 6 mois un de nos enquêteurs journalistes avec une amende très conséquente de plus de 60~000 couronnes finlandaises, nous avançons cette fois-ci des preuves solides et irréfutables, nous ne craignons aucune mise en examen, confiants que nous sommes sur l'issu de toute sorte de procès qui pourrait nous être intenté suite à la publication de ce numéro spécial.
		
		Comme d'habitude cher lecteur et chère lectrice, nous vous avons concocté le meilleur afin de vous régaler. Ce numéro regorge de ressources qui seront à votre goût, qui seront brutes, comme vous avez l'habitude, et, plus encore, les vilains petits secrets.
	\end{abstract}

\end{titlepage}

\tableofcontents

\newpage